\documentclass[final,hyperref={pdfpagelabels=false}]{beamer}
\usepackage{grffile}
\mode<presentation>{\usetheme{Rochester}\usecolortheme[RGB={140,15,45}]{structure}}
\usepackage[english]{babel}
\usepackage{graphicx}
\usepackage[latin1]{inputenc}
\usepackage{graphicx}
\usepackage{caption}
\usepackage[labelformat=empty]{subfig}
\usepackage{amsmath, amsthm, amssymb} 
%\usepackage{subcaption}
\beamertemplatenavigationsymbolsempty
\addto\captionsenglish{\renewcommand{\tablename}{Figure}}
\usepackage{amsmath,amsthm, amssymb, latexsym}
%\usepackage{times}\usefonttheme{professionalfonts}  % obsolete
%\usefonttheme[onlymath]{serif}
\boldmath
\usepackage[orientation=landscape,size=a0,scale=1.4,debug]{beamerposter}
% change list indention level
% \setdefaultleftmargin{3em}{}{}{}{}{}

%\usepackage{snapshot} % will write a .dep file with all dependencies, allows for easy bundling

\usepackage{array,booktabs,tabularx}
\newcolumntype{Z}{>{\centering\arraybackslash}X} % centered tabularx columns
\newcommand{\pphantom}{\textcolor{white}} % phantom introduces a vertical space in p formatted table columns??!!

\listfiles

%%%%%%%%%%%%%%%%%%%%%%%%%%%%%%%%%%%%%%%%%%%%%%%%%%%%%%%%%%%%%%%%%%%%%%%%%%%%%%%%%%%%%%
\graphicspath{{figures/}}
\title{\huge Plankton Classification with Deep Convolutional Neural Networks}
\author{Hendrik Weideman, Chuck Stewart}
\institute{Department of Computer Science\\Rensselaer Polytechnic Institute}
\date[April 2015]{April 2015}

%%%%%%%%%%%%%%%%%%%%%%%%%%%%%%%%%%%%%%%%%%%%%%%%%%%%%%%%%%%%%%%%%%%%%%%%%%%%%%%%%%%%%%
\newlength{\columnheight}
\setlength{\columnheight}{105cm}


%%%%%%%%%%%%%%%%%%%%%%%%%%%%%%%%%%%%%%%%%%%%%%%%%%%%%%%%%%%%%%%%%%%%%%%%%%%%%%%%%%%%%%
\begin{document}
\begin{frame}
  \begin{columns}[T]
  	\begin{column}{.32\textwidth}
      \begin{block}{Problem Statement}
        The abundance of different plankton species in lakes serves as a key indicator to determine the health of an aquatic ecosystem.
        Unfortunately, collecting and classifying these organisms by species is a challenging task, even for trained experts.  Thus,
        to determine the distribution of plankton across a lake, experts have to spend tedious hours manually classifying plankton by
        species.  We wish to streamline this process by developing a system that is capable of automatically classifying these organisms.
        By doing so, we aim to allow biologists to classify organisms at rates that are currently impossible because of the
        time-consuming nature of the task.  This would allow them to study the distribution of millions of plankton across a lake.  In
        addition, these distributions could be studied at multiple depths and at different points in time to provide a spatial-temporal
        map of species across a lake.  This information could provide biologists with novel insights into aquatic ecosystems on a scale
        that is currently infeasible.  Additionally, by collaborating with researchers from the arts and simulation sciences, we hope to
        use our system to allow them to develop a new interactive visualization of the plankton populations present in a lake.
      \end{block}

      \begin{block}{Data}
        We are currently collecting data to build a plankton dataset from the species present in Lake George.  Eventually this data
        will form the core of our training data.  Until this dataset is complete, however, we are working with a very recent dataset
        made available through a Kaggle competition to develop our classification algorithms.
        \begin{figure}
          \begin{minipage}{.5\textwidth}
            \centering
          \subfloat{\includegraphics{images/kaggle-data/acantharia_protist/38197.jpg}}
          \subfloat{\includegraphics{images/kaggle-data/acantharia_protist/36438.jpg}}
          \subfloat{\includegraphics{images/kaggle-data/acantharia_protist/78974.jpg}}
          \subfloat{\includegraphics{images/kaggle-data/acantharia_protist/98674.jpg}}
          \subfloat{\includegraphics{images/kaggle-data/acantharia_protist/97096.jpg}}

          \subfloat{\includegraphics{images/kaggle-data/acantharia_protist_big_center/105690.jpg}} 
          \subfloat{\includegraphics{images/kaggle-data/acantharia_protist_big_center/111204.jpg}} 
          \subfloat{\includegraphics{images/kaggle-data/acantharia_protist_big_center/48916.jpg}} 
          \subfloat{\includegraphics{images/kaggle-data/acantharia_protist_big_center/55072.jpg}} 
          \subfloat{\includegraphics{images/kaggle-data/acantharia_protist_big_center/36700.jpg}} 

          \subfloat{\includegraphics{images/kaggle-data/copepod_calanoid/56396.jpg}} 
          \subfloat{\includegraphics{images/kaggle-data/copepod_calanoid/80050.jpg}} 
          \subfloat{\includegraphics{images/kaggle-data/copepod_calanoid/76181.jpg}} 
          \subfloat{\includegraphics{images/kaggle-data/copepod_calanoid/96737.jpg}} 
          \subfloat{\includegraphics{images/kaggle-data/copepod_calanoid/97438.jpg}} 

          \subfloat{\includegraphics{images/kaggle-data/copepod_calanoid_flatheads/61098.jpg}} 
          \subfloat{\includegraphics{images/kaggle-data/copepod_calanoid_flatheads/61374.jpg}} 
          \subfloat{\includegraphics{images/kaggle-data/copepod_calanoid_flatheads/59563.jpg}} 
          \subfloat{\includegraphics{images/kaggle-data/copepod_calanoid_flatheads/90283.jpg}} 
          \subfloat{\includegraphics{images/kaggle-data/copepod_calanoid_flatheads/111088.jpg}} 

          \subfloat{\includegraphics{images/kaggle-data/hydromedusae_typeD/101612.jpg}} 
          \subfloat{\includegraphics{images/kaggle-data/hydromedusae_typeD/104800.jpg}} 
          \subfloat{\includegraphics{images/kaggle-data/hydromedusae_typeD/106211.jpg}} 
          \subfloat{\includegraphics{images/kaggle-data/hydromedusae_typeD/4939.jpg}} 
          \subfloat{\includegraphics{images/kaggle-data/hydromedusae_typeD/9019.jpg}} 

          \subfloat{\includegraphics{images/kaggle-data/hydromedusae_typeE/29026.jpg}} 
          \subfloat{\includegraphics{images/kaggle-data/hydromedusae_typeE/61896.jpg}} 
          \subfloat{\includegraphics{images/kaggle-data/hydromedusae_typeE/9348.jpg}} 
          \subfloat{\includegraphics{images/kaggle-data/hydromedusae_typeE/74042.jpg}} 
          \subfloat{\includegraphics{images/kaggle-data/hydromedusae_typeE/52215.jpg}} 
 
        \end{minipage}%
        \begin{minipage}{.5\textwidth}
          \centering
          \subfloat{\includegraphics{images/lake-george-data/copepod_calanoid/562.png}}
          \subfloat{\includegraphics{images/lake-george-data/copepod_calanoid/563.png}}
          \subfloat{\includegraphics{images/lake-george-data/copepod_calanoid/564.png}}
          \subfloat{\includegraphics{images/lake-george-data/copepod_calanoid/565.png}}
          \subfloat{\includegraphics{images/lake-george-data/copepod_calanoid/566.png}}

          \subfloat{\includegraphics{images/lake-george-data/copepod_cyclopoid/632.png}} 
          \subfloat{\includegraphics{images/lake-george-data/copepod_cyclopoid/633.png}} 
          \subfloat{\includegraphics{images/lake-george-data/copepod_cyclopoid/634.png}} 
          \subfloat{\includegraphics{images/lake-george-data/copepod_cyclopoid/635.png}} 
          \subfloat{\includegraphics{images/lake-george-data/copepod_cyclopoid/636.png}} 

          \subfloat{\includegraphics{images/lake-george-data/daphnia/702.png}} 
          \subfloat{\includegraphics{images/lake-george-data/daphnia/703.png}} 
          \subfloat{\includegraphics{images/lake-george-data/daphnia/704.png}} 
          \subfloat{\includegraphics{images/lake-george-data/daphnia/705.png}} 
          \subfloat{\includegraphics{images/lake-george-data/daphnia/707.png}} 

          \subfloat{\includegraphics{images/lake-george-data/holopedium/1000.png}} 
          \subfloat{\includegraphics{images/lake-george-data/holopedium/1001.png}} 
          \subfloat{\includegraphics{images/lake-george-data/holopedium/1002.png}} 
          \subfloat{\includegraphics{images/lake-george-data/holopedium/1003.png}} 
          \subfloat{\includegraphics{images/lake-george-data/holopedium/1004.png}} 

          \subfloat{\includegraphics{images/lake-george-data/diatom_penate/766.png}} 
          \subfloat{\includegraphics{images/lake-george-data/diatom_penate/767.png}} 
          \subfloat{\includegraphics{images/lake-george-data/diatom_penate/768.png}} 
          \subfloat{\includegraphics{images/lake-george-data/diatom_penate/769.png}} 
          \subfloat{\includegraphics{images/lake-george-data/diatom_penate/770.png}} 

          \subfloat{\includegraphics{images/lake-george-data/nauplii/1028.png}} 
          \subfloat{\includegraphics{images/lake-george-data/nauplii/1029.png}} 
          \subfloat{\includegraphics{images/lake-george-data/nauplii/1030.png}} 
          \subfloat{\includegraphics{images/lake-george-data/nauplii/1031.png}} 
          \subfloat{\includegraphics{images/lake-george-data/nauplii/1032.png}} 

        \end{minipage}
        \end{figure}	

      \end{block}
    \end{column}

  	\begin{column}{.32\textwidth}
      \begin{block}{The Learning Model}
      \end{block}
    \end{column}

  	\begin{column}{.32\textwidth}
      \begin{block}{The Future}
      \end{block}
    \end{column}
  \end{columns}
  \vskip1ex
  %\tiny\hfill\textcolor{ta2gray}{Created with \LaTeX \texttt{beamerposter}  \url{http://www-i6.informatik.rwth-aachen.de/~dreuw/latexbeamerposter.php}}
  \tiny\hfill{Created with \LaTeX \texttt{beamerposter}  \url{http://www-i6.informatik.rwth-aachen.de/~dreuw/latexbeamerposter.php} \hskip1em}
\end{frame}
\end{document}


%%%%%%%%%%%%%%%%%%%%%%%%%%%%%%%%%%%%%%%%%%%%%%%%%%%%%%%%%%%%%%%%%%%%%%%%%%%%%%%%%%%%%%%%%%%%%%%%%%%%
%%% Local Variables: 
%%% mode: latex
%%% TeX-PDF-mode: t
%%% End:
