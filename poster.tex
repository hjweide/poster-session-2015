\documentclass[final,hyperref={pdfpagelabels=false}]{beamer}
\usepackage{grffile}
\mode<presentation>{\usetheme{Rochester}\usecolortheme[RGB={140,15,45}]{structure}}
\usepackage[english]{babel}
\usepackage{graphicx}
\usepackage[latin1]{inputenc}
\usepackage{graphicx}
\usepackage{caption}
\usepackage[labelformat=empty]{subfig}
\usepackage{amsmath, amsthm, amssymb} 
%\usepackage{subcaption}
\beamertemplatenavigationsymbolsempty
\addto\captionsenglish{\renewcommand{\tablename}{Figure}}
\usepackage{amsmath,amsthm, amssymb, latexsym}
%\usepackage{times}\usefonttheme{professionalfonts}  % obsolete
%\usefonttheme[onlymath]{serif}
\boldmath
\usepackage[orientation=landscape,size=a0,scale=1.4,debug]{beamerposter}
% change list indention level
% \setdefaultleftmargin{3em}{}{}{}{}{}

%\usepackage{snapshot} % will write a .dep file with all dependencies, allows for easy bundling

\usepackage{array,booktabs,tabularx}
\newcolumntype{Z}{>{\centering\arraybackslash}X} % centered tabularx columns
\newcommand{\pphantom}{\textcolor{white}} % phantom introduces a vertical space in p formatted table columns??!!

\listfiles

%%%%%%%%%%%%%%%%%%%%%%%%%%%%%%%%%%%%%%%%%%%%%%%%%%%%%%%%%%%%%%%%%%%%%%%%%%%%%%%%%%%%%%
\graphicspath{{figures/}}
\title{\huge Plankton Classification with Deep Convolutional Neural Networks}
\author{Hendrik Weideman, Chuck Stewart}
\institute{Department of Computer Science\\Rensselaer Polytechnic Institute}
\date[April 2015]{April 2015}

%%%%%%%%%%%%%%%%%%%%%%%%%%%%%%%%%%%%%%%%%%%%%%%%%%%%%%%%%%%%%%%%%%%%%%%%%%%%%%%%%%%%%%
\newlength{\columnheight}
\setlength{\columnheight}{105cm}


%%%%%%%%%%%%%%%%%%%%%%%%%%%%%%%%%%%%%%%%%%%%%%%%%%%%%%%%%%%%%%%%%%%%%%%%%%%%%%%%%%%%%%
\begin{document}
\begin{frame}
  \begin{columns}[T]
  	\begin{column}{.32\textwidth}
      \begin{block}{Problem Statement}
        The distribution of different plankton species in a lake serves as a key indicator to determine the health of an aquatic
        ecosystem.  Unfortunately, collecting and classifying these organisms is a challenging task, even for highly-trained
        experts.
        We hope to make this process easier by developing a system that can automatically classify organisms into different classes of
        plankton.  Ideally this would help biologists to
        \begin{itemize}
          \item study the distribution of millions of organisms across a lake,
          \item study how these distributions change at different depths and locations, and
          \item build a spatial-temporal map of species across a lake.
        \end{itemize}
      \end{block}

      \begin{block}{Project Goals}
        Currently, such a study would be infeasible because of the sheer effort required to classify organisms on this scale.
        Thus, we aim to streamline this process to the point where such a study could be practical, by developing a learning model
        that can
        \begin{itemize}
          \item automatically classify plankton into different pre-defined classes,
          \item be used to assist a human operator to accelerate the process of manual annotation, and
          \item be extended to recognize new classes without prohibitively expensive re-training.
        \end{itemize}
      \end{block}

      \begin{block}{The Data: Background}
        We are currently collecting data to build an annotated plankton dataset from the species present in Lake George. 
        Eventually this data
        will form the core of our training data.  Until this dataset is complete, however, we are working with a very recent dataset
        made available through a Kaggle competition to develop our classification algorithms.  Below we show images from both datasets.
        On the left are images from the Kaggle dataset, and on the right are some images from the in-progress Lake George dataset.
        All the organisms in the same row belong to the same class.

        \begin{figure}
          \begin{minipage}{.5\textwidth}
            \centering
          \subfloat{\includegraphics{images/kaggle-data/acantharia_protist/38197.jpg}}
          \subfloat{\includegraphics{images/kaggle-data/acantharia_protist/36438.jpg}}
          \subfloat{\includegraphics{images/kaggle-data/acantharia_protist/78974.jpg}}
          \subfloat{\includegraphics{images/kaggle-data/acantharia_protist/98674.jpg}}
          \subfloat{\includegraphics{images/kaggle-data/acantharia_protist/97096.jpg}}

          \subfloat{\includegraphics{images/kaggle-data/acantharia_protist_big_center/105690.jpg}} 
          \subfloat{\includegraphics{images/kaggle-data/acantharia_protist_big_center/111204.jpg}} 
          \subfloat{\includegraphics{images/kaggle-data/acantharia_protist_big_center/48916.jpg}} 
          \subfloat{\includegraphics{images/kaggle-data/acantharia_protist_big_center/55072.jpg}} 
          \subfloat{\includegraphics{images/kaggle-data/acantharia_protist_big_center/36700.jpg}} 

          \subfloat{\includegraphics{images/kaggle-data/copepod_calanoid/56396.jpg}} 
          \subfloat{\includegraphics{images/kaggle-data/copepod_calanoid/80050.jpg}} 
          \subfloat{\includegraphics{images/kaggle-data/copepod_calanoid/76181.jpg}} 
          \subfloat{\includegraphics{images/kaggle-data/copepod_calanoid/96737.jpg}} 
          \subfloat{\includegraphics{images/kaggle-data/copepod_calanoid/97438.jpg}} 

          \subfloat{\includegraphics{images/kaggle-data/copepod_calanoid_flatheads/61098.jpg}} 
          \subfloat{\includegraphics{images/kaggle-data/copepod_calanoid_flatheads/61374.jpg}} 
          \subfloat{\includegraphics{images/kaggle-data/copepod_calanoid_flatheads/59563.jpg}} 
          \subfloat{\includegraphics{images/kaggle-data/copepod_calanoid_flatheads/90283.jpg}} 
          \subfloat{\includegraphics{images/kaggle-data/copepod_calanoid_flatheads/111088.jpg}} 

          \subfloat{\includegraphics{images/kaggle-data/hydromedusae_typeD/101612.jpg}} 
          \subfloat{\includegraphics{images/kaggle-data/hydromedusae_typeD/104800.jpg}} 
          \subfloat{\includegraphics{images/kaggle-data/hydromedusae_typeD/106211.jpg}} 
          \subfloat{\includegraphics{images/kaggle-data/hydromedusae_typeD/4939.jpg}} 
          \subfloat{\includegraphics{images/kaggle-data/hydromedusae_typeD/9019.jpg}} 

          \subfloat{\includegraphics{images/kaggle-data/hydromedusae_typeE/29026.jpg}} 
          \subfloat{\includegraphics{images/kaggle-data/hydromedusae_typeE/61896.jpg}} 
          \subfloat{\includegraphics{images/kaggle-data/hydromedusae_typeE/9348.jpg}} 
          \subfloat{\includegraphics{images/kaggle-data/hydromedusae_typeE/74042.jpg}} 
          \subfloat{\includegraphics{images/kaggle-data/hydromedusae_typeE/52215.jpg}} 
 
        \end{minipage}%
        \begin{minipage}{.5\textwidth}
          \centering
          \subfloat{\includegraphics{images/lake-george-data/copepod_calanoid/562.png}}
          \subfloat{\includegraphics{images/lake-george-data/copepod_calanoid/563.png}}
          \subfloat{\includegraphics{images/lake-george-data/copepod_calanoid/564.png}}
          \subfloat{\includegraphics{images/lake-george-data/copepod_calanoid/565.png}}
          \subfloat{\includegraphics{images/lake-george-data/copepod_calanoid/566.png}}

          \subfloat{\includegraphics{images/lake-george-data/copepod_cyclopoid/632.png}} 
          \subfloat{\includegraphics{images/lake-george-data/copepod_cyclopoid/633.png}} 
          \subfloat{\includegraphics{images/lake-george-data/copepod_cyclopoid/634.png}} 
          \subfloat{\includegraphics{images/lake-george-data/copepod_cyclopoid/635.png}} 
          \subfloat{\includegraphics{images/lake-george-data/copepod_cyclopoid/636.png}} 

          \subfloat{\includegraphics{images/lake-george-data/daphnia/702.png}} 
          \subfloat{\includegraphics{images/lake-george-data/daphnia/703.png}} 
          \subfloat{\includegraphics{images/lake-george-data/daphnia/704.png}} 
          \subfloat{\includegraphics{images/lake-george-data/daphnia/705.png}} 
          \subfloat{\includegraphics{images/lake-george-data/daphnia/707.png}} 

          \subfloat{\includegraphics{images/lake-george-data/holopedium/1000.png}} 
          \subfloat{\includegraphics{images/lake-george-data/holopedium/1001.png}} 
          \subfloat{\includegraphics{images/lake-george-data/holopedium/1002.png}} 
          \subfloat{\includegraphics{images/lake-george-data/holopedium/1003.png}} 
          \subfloat{\includegraphics{images/lake-george-data/holopedium/1004.png}} 

          \subfloat{\includegraphics{images/lake-george-data/diatom_penate/766.png}} 
          \subfloat{\includegraphics{images/lake-george-data/diatom_penate/767.png}} 
          \subfloat{\includegraphics{images/lake-george-data/diatom_penate/768.png}} 
          \subfloat{\includegraphics{images/lake-george-data/diatom_penate/769.png}} 
          \subfloat{\includegraphics{images/lake-george-data/diatom_penate/770.png}} 

          \subfloat{\includegraphics{images/lake-george-data/nauplii/1028.png}} 
          \subfloat{\includegraphics{images/lake-george-data/nauplii/1029.png}} 
          \subfloat{\includegraphics{images/lake-george-data/nauplii/1030.png}} 
          \subfloat{\includegraphics{images/lake-george-data/nauplii/1031.png}} 
          \subfloat{\includegraphics{images/lake-george-data/nauplii/1032.png}} 

        \end{minipage}
        \end{figure}	

      \end{block}
    \end{column}

  	\begin{column}{.32\textwidth}
      \begin{block}{The Data: Challenges}
        \begin{itemize}
          \item Two organisms from the same species do not necessarily belong to the same class.  Often a species-level classification
            is too coarse for biologists when studying the complex dynamics in an ecosystem, and so our learning model
            needs to deal with some classes that are distinct, yet visually very similar.
          \item The images are two-dimensional representations of a three-dimensional world, and so organisms appear in various
            orientations around three rotational axes.
        \end{itemize}
      \end{block}

      \begin{block}{A Simple Deep Convolutional Neural Network}
        \begin{figure}
          \includegraphics[width=0.9\textwidth]{images/svg/convnet.png}
        \end{figure}
      \end{block}
    \end{column}

  	\begin{column}{.32\textwidth}
      \begin{block}{A Stack of Non-Linear Layers}
        The deep convolutional neural network shown on the left has been greatly simplified for visualization purposes, with only
        two feature maps per layer and only a single convolutional and pooling layer, respectively.  Additionally, many connections
        between layers are omitted for visual clarity.
        A real network would typically have several convolutional and pooling layers, with anything from tens to hundreds of feature
        maps per layer.  Additionally, the output from each convolutional and fully-connected layer typically passes through a 
        non-linear function called the \emph{activation function}.  This stack of layers, each with its own non-linearity, is what
        allows deep convolutional neural networks to learn a highly non-linear mapping from input to output.
      \end{block}

      \begin{block}{Different Layers for Different Purposes}
        \begin{itemize} 
          \item The \textbf{input layer} takes a gray scale image, or color image separated into channels, and processes 
            the raw pixel data directly, with an optional pre-processing transformation.
          \item \textbf{Convolutional layers} consist of multiple convolutional feature maps, where each feature map has an associated
            convolutional kernel that is learned by means of error backpropagation.  This kernel is replicated over the entire input
            with shared weight parameters.
          \item \textbf{Pooling layers} introduce a degree of translational invariance by selectively keeping outputs from the
            preceding convolutional layer's feature maps.  When only the maximum responses are kept, this is so-called
            \emph{max-pooling}.
          \item \textbf{Fully-connected layers} are closely related to the hidden layers of a standard artificial neural network.  They
            disregard any spatial information from the previous layer and simply compute a function mapping the output of the 
            previous layer to target outputs.
          \item The \textbf{output layer} is typically another fully-connected layer with a special non-linearity, typically a 
            \emph{softmax} activation function.  This layer computes a probability distribution function over the target classes
            from the output of the previous layer.
        \end{itemize} 
      \end{block}

      \begin{block}{The Future}
        Once our system is complete, we hope that it may be be used to
        \begin{itemize}
          \item produce new annotated datasets for scientific research,
          \item accelerate manual annotation to allow researchers to label their own datasets,
          \item assist biologists in gaining a better understanding of complex aquatic ecosystems, and
          \item assist researchers in the simulation sciences to create an immersive visualization of plankton communities.
        \end{itemize}
      \end{block}
      \includegraphics[width=\textwidth]{images/logo.eps}
    \end{column}
  \end{columns}
  \vskip1ex
  %\tiny\hfill\textcolor{ta2gray}{Created with \LaTeX \texttt{beamerposter}  \url{http://www-i6.informatik.rwth-aachen.de/~dreuw/latexbeamerposter.php}}
  \tiny\hfill{Created with \LaTeX \texttt{beamerposter}  \url{http://www-i6.informatik.rwth-aachen.de/~dreuw/latexbeamerposter.php} \hskip1em}
\end{frame}
\end{document}


%%%%%%%%%%%%%%%%%%%%%%%%%%%%%%%%%%%%%%%%%%%%%%%%%%%%%%%%%%%%%%%%%%%%%%%%%%%%%%%%%%%%%%%%%%%%%%%%%%%%
%%% Local Variables: 
%%% mode: latex
%%% TeX-PDF-mode: t
%%% End:
